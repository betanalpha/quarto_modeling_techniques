\documentclass{standalone}
\usepackage{graphicx}	
\usepackage{amssymb, amsmath}
\usepackage{color}

\usepackage{tikz}
\usetikzlibrary{intersections, backgrounds, math, arrows.meta}
\usepackage{pgfmath}

\definecolor{light}{RGB}{220, 188, 188}
\definecolor{mid}{RGB}{185, 124, 124}
\definecolor{dark}{RGB}{143, 39, 39}
\definecolor{highlight}{RGB}{180, 31, 180}
\definecolor{gray10}{gray}{0.1}
\definecolor{gray20}{gray}{0.2}
\definecolor{gray30}{gray}{0.3}
\definecolor{gray40}{gray}{0.4}
\definecolor{gray60}{gray}{0.6}
\definecolor{gray70}{gray}{0.7}
\definecolor{gray80}{gray}{0.8}
\definecolor{gray90}{gray}{0.9}
\definecolor{gray95}{gray}{0.95}

\tikzmath{
  function normal(\x, \m, \s) {
    return exp(-0.5 * (\x - \m) * (\x - \m) / (\s * \s) ) / (2.506628274631001 * \s);
  };
}

\begin{document}

\begin{tikzpicture}[scale=1]

  \begin{scope}[shift={(0, 0)}]
    \draw[white] (-4.5, -3) rectangle (3.5, 2.5);
    
    \fill[gray95] (-2, -2) rectangle (-0.75, 2);
    \fill[gray95] (0.5, -2) rectangle (1.75, 2);
  
    \begin{scope}
      \clip (-3, -2.1) rectangle (2.85, 2);
      
      \draw[domain=-3:3, smooth, samples=100, variable=\x, mid, line width=1] 
        plot (\x, { 4 * ( 0.5 * normal(\x, -1.375, 0.2693966) ) - 2});
        
      \draw[domain=-3:3, smooth, samples=100, variable=\x, light, line width=1] 
        plot (\x, { 4 * ( 0.5 * normal(\x, 1.125, 0.2693966) ) - 2});
        
      \draw[domain=-3:3, smooth, samples=100, variable=\x, dark, line width=1] 
       plot (\x, { 4 * ( 0.5 * normal(\x, -1.375, 0.2693966) + 0.5 * normal(\x, 1.125, 0.2693966) ) - 2}); 
    \end{scope}
    
    \draw [->, >=stealth, line width=1] (-3.00, -2.015) -- +(0, 4);
    \draw [->, >=stealth, line width=1] (-3.015, -2.00) -- +(6, 0);
    
    \draw[arrows={Bracket[sharp]-Bracket[sharp]}, gray70, line width=1] (-2, -2) -- (-0.75, -2);
    \draw[arrows={Bracket[sharp]-Bracket[sharp]}, gray70, line width=1] (0.5, -2) -- (1.75, -2);

    \node at (-3.75, 0) { $p(\theta)$ };
    \node at (0, -2.5) { $\theta$ };
    
  \end{scope}

\end{tikzpicture}

\end{document}  